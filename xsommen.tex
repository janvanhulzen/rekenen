\documentclass{article}
\usepackage{color}
\usepackage[usenames,dvipsnames,svgnames,table]{xcolor}
\usepackage[dutch]{babel}
\usepackage{caption}
%\usepackage{xspace,colortbl}
%\usepackage{a4wide}
\usepackage{graphicx}
\usepackage{fancyhdr}
\usepackage{latexsym}
\usepackage{amsmath,amsfonts}
\usepackage{array}
\usepackage{soul} 
\usepackage{array}

\usepackage[normalem]{ulem}               % to striketrhourhg text
\newcommand\redout{\bgroup\markoverwith
{\textcolor{red}{\rule[0.5ex]{2pt}{2pt}}}\ULon}

\newcolumntype{M}[1]{>{\centering\arraybackslash}m{#1}}
\newcolumntype{N}{@{}m{0pt}@{}}

\usepackage{framed}
\topmargin=-2cm
\textheight=24cm
\textwidth=17cm
\footskip=0.5cm
\hoffset=-32mm

%Laplace table defines

\usepackage[latin1]{inputenc}
%\usepackage[left=.5in,right=.5in,top=.5in,bottom=.5in]{geometry} 

\usepackage{amssymb}
\usepackage{setspace}
\usepackage[colorlinks=true]{hyperref}
\usepackage{textcomp} 
\usepackage{multicol} 
 
\newcounter{NumberInTable}
\newcommand{\LTNUM}{\stepcounter{NumberInTable}{(\theNumberInTable)}}





\newcommand{\Laplace}[1]{\ensuremath{\mathcal{L}{\left\{#1\right\}}}}
\newcommand{\InvLap}[1]{\ensuremath{\mathcal{L}^{-1}{\left\{#1\right\}}}}

\makeatletter
\@addtoreset{section}{part}
\makeatother
\newcounter{vraagnummer}
\setcounter{vraagnummer}{1}
\newcounter{itemcounter}
\setcounter{itemcounter}{0}



\usepackage{fancyhdr}
\setlength{\headheight}{15pt}
 
\pagestyle{fancyplain}
%\renewcommand{\chaptermark}[1]{\markboth{#1}{bla}}
 
%\lhead{\fancyplain{}{\thepage}}
\chead{X-sommen om mee te oefenen}
\lhead{}
\rhead{\thepage /11}
\lfoot{}
\cfoot{}
\rfoot{}

\makeatletter
\@addtoreset{section}{part}
\makeatother

\begin{document}
\noindent Onderstaande sommen maak je door de rijtjes in te vullen. Hieronder staat een voorbeeld, 57 x 456 . Je zet eerst de getallen onder elkaar, met het kleinste getal beneden:\newline

\begin{tabular}{llllll}
&&4&5&\color{orange}6\color{black}&\\
&&&5&\color{orange}7\color{black}&x\\
\hline
\end{tabular}\newline
\newline
\newline
De eerste stap is \color{orange}7\color{black}x\color{orange}6\color{black}=42. Je schrijft 2 op en onthoudt \color{red}4\color{black}.\newline
\newline
\begin{tabular}{llllll}
&&&\color{red}4\color{black}&&\\
&&4&5\color{black}&\color{orange}6&\\
&&&5&\color{orange}7\color{black}&x\\
\hline
&&&&2&\\
\end{tabular}\newline
\newline
onthouden doe je door de \color{red} 4 \color{black} boven de 5 te zetten. De volgende stap is \color{orange}7\color{black}x\color{orange}5\color{black}=35. De \color{red} 4 \color{black} die je net hebt onthouden tel je daar bij op. Je krijgt  \color{orange}7\color{black}x\color{orange}5\color{black}=35+ \color{red} 4 \color{black} = 39. Schrijf vervolgens de 9 op en onthoud de  \color{blue} 3 \color{black}. \newline
\newline
\begin{tabular}{llllll}
&& \color{blue}3\color{black}&\color{red}4&&\\
&&4&\color{orange}5&6&\\
&&&5&\color{orange}7&x\\
\hline
&&&9&2&\\
\end{tabular}\newline
\newline
Nu volgt de laatste stap van het eerste rijtje. We hebben \color{orange}7\color{black}x\color{orange}4\color{black}=28 en omdat we  \color{blue} 3 \color{black} hebben onthouden volgt \color{orange}7\color{black}x\color{orange}4\color{black}=28+ \color{blue} 3 \color{black}=31. Er zijn geen getallen meer in het bovenste rijtje en je kan 31 helemaal opschrijven. \newline 
\newline
\begin{tabular}{llllll}
&&\color{blue}3&4&&\\
&&\color{orange}4\color{black}&5&6&\\
&&&5&\color{orange}7\color{black}&x\\
\hline
&3&1&9&2&\\
\end{tabular}\newline
\newline
De eerste rij is klaar! Op de tweede rij begin je met het opschrijven van een '\color{red}0\color{black}' en je werkt daarna op dezelfde manier de som af. Je kunt eventueel een streepje door de getallen zetten die je hebt onthouden.
\newline
\newline
\begin{tabular}{llllll}
&&\redout{3}&\redout{4}&&\\
&&4&5&6&\\
&&&5&7&x\\
\hline
&3&1&9&2&\\
&&&&\color{red}0\color{black}&\\
\end{tabular}\newline
\newline
De volgende stappen gaan zo: 5x6=30, nul opschrijven en 3 onthouden. \newline Dan 5x5=25, we hebben 3 onthouden dus dat wordt 5x5=25+3=28. Je schrijft \color{blue} 8 \color{black} op en onthoudt 2. \newline De laatste stap is 5x4=20 waarbij we 2 onthouden hebben zodat volgt 6x4=20+2=22. Omdat het rijtje klaar is kan je \color{blue} 22 \color{black} direct opschrijven.
\newline
\newline
\begin{tabular}{llllll}
&&2&3&&\\
&&\redout{3}&\redout{4}&&\\
&&\color{orange}4\color{black}&\color{orange}5\color{black}&\color{orange}6\color{black}&\\
&&&\color{orange}5\color{black}&7&x\\
\hline
&3&1&9&2&\\
\color{blue} 2 \color{black}&\color{blue} 2 \color{black}&\color{blue} 8 \color{black}&0&\color{red}0\color{black}&\\
\end{tabular}\newline
\newline
De laatste stap is het optellen van de twee onderste rijtjes
\newline
\newline
\begin{tabular}{llllll}
&&\redout{2}&\redout{3}&&\\
&&\redout{3}&\redout{4}&&\\
&&4&5&6&\\
&&&5&7&x\\
\hline
&\color{orange}3&\color{orange}1&\color{orange}9&\color{orange}2&\\
\color{orange}2&\color{orange}2&\color{orange}8&\color{orange}0&\color{orange}0&+\\
\hline
2&5&9&9&2&\\
\end{tabular}\newline
\newline
Klaar! Vergeet niet dat je bij optellen  soms ook getallen moet onthouden.
%\begin{tabular}{llllll}
%&&4&5&6&\\
%&&&5&7&x\\
%\hline
%2&5&9&9&2&\\
%\end{tabular}

\Large
\input{xsommen_antwoorden.tex}

\begin{tabular}{l c r }
\vspace{3cm}

\begin{tabular}{llllll}
&&4&1&9&\\
&&&7&4&x\\
\hline
3&1&0&0&6&\\
\end{tabular}&
\begin{tabular}{llllll}
&&5&6&7&\\
&&&9&7&x\\
\hline
5&4&9&9&9&\\
\end{tabular}&
\begin{tabular}{llllll}
&&6&4&7&\\
&&&4&8&x\\
\hline
3&1&0&5&6&\\
\end{tabular}\\\vspace{3cm}
\begin{tabular}{llllll}
&&7&3&4&\\
&&&3&9&x\\
\hline
2&8&6&2&6&\\
\end{tabular}&
\begin{tabular}{llllll}
&&4&5&1&\\
&&&9&8&x\\
\hline
4&4&1&9&8&\\
\end{tabular}&
\begin{tabular}{llllll}
&&3&9&6&\\
&&&5&6&x\\
\hline
2&2&1&7&6&\\
\end{tabular}\\\vspace{3cm}
\begin{tabular}{llllll}
&&7&4&5&\\
&&&6&3&x\\
\hline
4&6&9&3&5&\\
\end{tabular}&
\begin{tabular}{llllll}
&&3&7&4&\\
&&&8&9&x\\
\hline
3&3&2&8&6&\\
\end{tabular}&
\begin{tabular}{llllll}
&&1&5&9&\\
&&&9&7&x\\
\hline
1&5&4&2&3&\\
\end{tabular}\\\vspace{3cm}
\begin{tabular}{llllll}
&&9&8&6&\\
&&&1&7&x\\
\hline
1&6&7&6&2&\\
\end{tabular}&
\begin{tabular}{llllll}
&&6&5&8&\\
&&&9&8&x\\
\hline
6&4&4&8&4&\\
\end{tabular}&
\begin{tabular}{llllll}
&&1&5&3&\\
&&&4&5&x\\
\hline
 &6&8&8&5&\\
\end{tabular}\\\vspace{3cm}
\begin{tabular}{llllll}
&&6&5&4&\\
&&&5&8&x\\
\hline
3&7&9&3&2&\\
\end{tabular}&
\begin{tabular}{llllll}
&&5&9&7&\\
&&&8&5&x\\
\hline
5&0&7&4&5&\\
\end{tabular}&
\begin{tabular}{llllll}
&&8&9&7&\\
&&&1&8&x\\
\hline
1&6&1&4&6&\\
\end{tabular}\\\vspace{3cm}\end{tabular}\newpage
\begin{tabular}{l c r }
\vspace{3cm}

\begin{tabular}{llllll}
&&8&9&1&\\
&&&3&7&x\\
\hline
3&2&9&6&7&\\
\end{tabular}&
\begin{tabular}{llllll}
&&8&3&5&\\
&&&5&3&x\\
\hline
4&4&2&5&5&\\
\end{tabular}&
\begin{tabular}{llllll}
&&5&3&9&\\
&&&6&9&x\\
\hline
3&7&1&9&1&\\
\end{tabular}\\\vspace{3cm}
\begin{tabular}{llllll}
&&1&6&3&\\
&&&3&7&x\\
\hline
 &6&0&3&1&\\
\end{tabular}&
\begin{tabular}{llllll}
&&5&8&7&\\
&&&8&6&x\\
\hline
5&0&4&8&2&\\
\end{tabular}&
\begin{tabular}{llllll}
&&5&7&4&\\
&&&1&3&x\\
\hline
 &7&4&6&2&\\
\end{tabular}\\\vspace{3cm}
\begin{tabular}{llllll}
&&7&8&6&\\
&&&8&4&x\\
\hline
6&6&0&2&4&\\
\end{tabular}&
\begin{tabular}{llllll}
&&9&5&6&\\
&&&3&6&x\\
\hline
3&4&4&1&6&\\
\end{tabular}&
\begin{tabular}{llllll}
&&1&5&6&\\
&&&7&9&x\\
\hline
1&2&3&2&4&\\
\end{tabular}\\\vspace{3cm}
\begin{tabular}{llllll}
&&9&8&4&\\
&&&5&1&x\\
\hline
5&0&1&8&4&\\
\end{tabular}&
\begin{tabular}{llllll}
&&5&3&8&\\
&&&5&9&x\\
\hline
3&1&7&4&2&\\
\end{tabular}&
\begin{tabular}{llllll}
&&3&6&9&\\
&&&7&4&x\\
\hline
2&7&3&0&6&\\
\end{tabular}\\\vspace{3cm}
\begin{tabular}{llllll}
&&6&5&4&\\
&&&9&7&x\\
\hline
6&3&4&3&8&\\
\end{tabular}&
\begin{tabular}{llllll}
&&4&3&6&\\
&&&1&5&x\\
\hline
 &6&5&4&0&\\
\end{tabular}&
\begin{tabular}{llllll}
&&6&3&4&\\
&&&9&5&x\\
\hline
6&0&2&3&0&\\
\end{tabular}\\\vspace{3cm}\end{tabular}\newpage
\begin{tabular}{l c r }
\vspace{3cm}

\begin{tabular}{llllll}
&&8&9&6&\\
&&&9&4&x\\
\hline
8&4&2&2&4&\\
\end{tabular}&
\begin{tabular}{llllll}
&&4&8&5&\\
&&&3&5&x\\
\hline
1&6&9&7&5&\\
\end{tabular}&
\begin{tabular}{llllll}
&&5&3&7&\\
&&&8&5&x\\
\hline
4&5&6&4&5&\\
\end{tabular}\\\vspace{3cm}
\begin{tabular}{llllll}
&&7&1&5&\\
&&&6&5&x\\
\hline
4&6&4&7&5&\\
\end{tabular}&
\begin{tabular}{llllll}
&&7&3&9&\\
&&&8&3&x\\
\hline
6&1&3&3&7&\\
\end{tabular}&
\begin{tabular}{llllll}
&&1&3&6&\\
&&&1&6&x\\
\hline
 &2&1&7&6&\\
\end{tabular}\\\vspace{3cm}
\begin{tabular}{llllll}
&&6&7&8&\\
&&&4&1&x\\
\hline
2&7&7&9&8&\\
\end{tabular}&
\begin{tabular}{llllll}
&&9&7&1&\\
&&&6&8&x\\
\hline
6&6&0&2&8&\\
\end{tabular}&
\begin{tabular}{llllll}
&&6&3&4&\\
&&&1&6&x\\
\hline
1&0&1&4&4&\\
\end{tabular}\\\vspace{3cm}
\begin{tabular}{llllll}
&&5&9&3&\\
&&&5&6&x\\
\hline
3&3&2&0&8&\\
\end{tabular}&
\begin{tabular}{llllll}
&&8&6&1&\\
&&&4&3&x\\
\hline
3&7&0&2&3&\\
\end{tabular}&
\begin{tabular}{llllll}
&&1&8&7&\\
&&&7&6&x\\
\hline
1&4&2&1&2&\\
\end{tabular}\\\vspace{3cm}
\begin{tabular}{llllll}
&&7&1&8&\\
&&&4&7&x\\
\hline
3&3&7&4&6&\\
\end{tabular}&
\begin{tabular}{llllll}
&&1&7&4&\\
&&&7&9&x\\
\hline
1&3&7&4&6&\\
\end{tabular}&
\begin{tabular}{llllll}
&&3&1&5&\\
&&&1&8&x\\
\hline
 &5&6&7&0&\\
\end{tabular}\\\vspace{3cm}\end{tabular}\newpage
\begin{tabular}{l c r }
\vspace{3cm}

\begin{tabular}{llllll}
&&3&8&1&\\
&&&8&9&x\\
\hline
3&3&9&0&9&\\
\end{tabular}&
\begin{tabular}{llllll}
&&4&3&9&\\
&&&8&3&x\\
\hline
3&6&4&3&7&\\
\end{tabular}&
\begin{tabular}{llllll}
&&6&8&9&\\
&&&1&9&x\\
\hline
1&3&0&9&1&\\
\end{tabular}\\\vspace{3cm}
\begin{tabular}{llllll}
&&6&3&5&\\
&&&3&5&x\\
\hline
2&2&2&2&5&\\
\end{tabular}&
\begin{tabular}{llllll}
&&4&8&3&\\
&&&7&1&x\\
\hline
3&4&2&9&3&\\
\end{tabular}&
\begin{tabular}{llllll}
&&7&9&5&\\
&&&9&4&x\\
\hline
7&4&7&3&0&\\
\end{tabular}\\\vspace{3cm}
\begin{tabular}{llllll}
&&5&4&7&\\
&&&3&5&x\\
\hline
1&9&1&4&5&\\
\end{tabular}&
\begin{tabular}{llllll}
&&8&5&7&\\
&&&8&9&x\\
\hline
7&6&2&7&3&\\
\end{tabular}&
\begin{tabular}{llllll}
&&1&7&3&\\
&&&3&5&x\\
\hline
 &6&0&5&5&\\
\end{tabular}\\\vspace{3cm}
\begin{tabular}{llllll}
&&7&1&8&\\
&&&3&6&x\\
\hline
2&5&8&4&8&\\
\end{tabular}&
\begin{tabular}{llllll}
&&4&5&8&\\
&&&9&3&x\\
\hline
4&2&5&9&4&\\
\end{tabular}&
\begin{tabular}{llllll}
&&3&7&6&\\
&&&1&8&x\\
\hline
 &6&7&6&8&\\
\end{tabular}\\\vspace{3cm}
\begin{tabular}{llllll}
&&5&8&6&\\
&&&4&1&x\\
\hline
2&4&0&2&6&\\
\end{tabular}&
\begin{tabular}{llllll}
&&1&6&4&\\
&&&4&7&x\\
\hline
 &7&7&0&8&\\
\end{tabular}&
\begin{tabular}{llllll}
&&6&1&5&\\
&&&8&1&x\\
\hline
4&9&8&1&5&\\
\end{tabular}\\\vspace{3cm}\end{tabular}\newpage
\begin{tabular}{l c r }
\vspace{3cm}

\begin{tabular}{llllll}
&&7&6&4&\\
&&&1&4&x\\
\hline
1&0&6&9&6&\\
\end{tabular}&
\begin{tabular}{llllll}
&&7&3&8&\\
&&&4&5&x\\
\hline
3&3&2&1&0&\\
\end{tabular}&
\begin{tabular}{llllll}
&&9&4&3&\\
&&&5&4&x\\
\hline
5&0&9&2&2&\\
\end{tabular}\\\vspace{3cm}
\begin{tabular}{llllll}
&&4&3&9&\\
&&&6&4&x\\
\hline
2&8&0&9&6&\\
\end{tabular}&
\begin{tabular}{llllll}
&&9&7&6&\\
&&&7&3&x\\
\hline
7&1&2&4&8&\\
\end{tabular}&
\begin{tabular}{llllll}
&&3&4&5&\\
&&&3&7&x\\
\hline
1&2&7&6&5&\\
\end{tabular}\\\vspace{3cm}
\begin{tabular}{llllll}
&&7&5&8&\\
&&&4&6&x\\
\hline
3&4&8&6&8&\\
\end{tabular}&
\begin{tabular}{llllll}
&&7&9&8&\\
&&&1&7&x\\
\hline
1&3&5&6&6&\\
\end{tabular}&
\begin{tabular}{llllll}
&&3&1&5&\\
&&&8&3&x\\
\hline
2&6&1&4&5&\\
\end{tabular}\\\vspace{3cm}
\begin{tabular}{llllll}
&&7&4&5&\\
&&&6&1&x\\
\hline
4&5&4&4&5&\\
\end{tabular}&
\begin{tabular}{llllll}
&&9&6&7&\\
&&&1&8&x\\
\hline
1&7&4&0&6&\\
\end{tabular}&
\begin{tabular}{llllll}
&&9&7&3&\\
&&&3&1&x\\
\hline
3&0&1&6&3&\\
\end{tabular}\\\vspace{3cm}
\begin{tabular}{llllll}
&&6&1&3&\\
&&&7&1&x\\
\hline
4&3&5&2&3&\\
\end{tabular}&
\begin{tabular}{llllll}
&&1&5&4&\\
&&&6&5&x\\
\hline
1&0&0&1&0&\\
\end{tabular}&
\begin{tabular}{llllll}
&&8&6&3&\\
&&&6&8&x\\
\hline
5&8&6&8&4&\\
\end{tabular}\\\vspace{3cm}\end{tabular}\newpage

\end{document}