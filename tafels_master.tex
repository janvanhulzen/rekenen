\documentclass{article}
\usepackage{color}
\usepackage[dutch]{babel}
\usepackage{caption}
%\usepackage{xspace,colortbl}
%\usepackage{a4wide}
\usepackage{graphicx}
\usepackage{fancyhdr}
\usepackage{latexsym}
\usepackage{amsmath,amsfonts}
\usepackage{array}
\newcolumntype{M}[1]{>{\centering\arraybackslash}m{#1}}
\newcolumntype{N}{@{}m{0pt}@{}}

\usepackage{framed}
\topmargin=-2cm
\textheight=24cm
\textwidth=17cm
\footskip=0.5cm
\hoffset=-32mm

%Laplace table defines

\usepackage[latin1]{inputenc}
%\usepackage[left=.5in,right=.5in,top=.5in,bottom=.5in]{geometry} 

\usepackage{amssymb}
\usepackage{setspace}
\usepackage[colorlinks=true]{hyperref}
\usepackage{textcomp} 
\usepackage{multicol} 
 
\newcounter{NumberInTable}
\newcommand{\LTNUM}{\stepcounter{NumberInTable}{(\theNumberInTable)}}





\newcommand{\Laplace}[1]{\ensuremath{\mathcal{L}{\left\{#1\right\}}}}
\newcommand{\InvLap}[1]{\ensuremath{\mathcal{L}^{-1}{\left\{#1\right\}}}}

\makeatletter
\@addtoreset{section}{part}
\makeatother
\newcounter{vraagnummer}
\setcounter{vraagnummer}{1}
\newcounter{itemcounter}
\setcounter{itemcounter}{0}



\usepackage{fancyhdr}
\setlength{\headheight}{15pt}
 
\pagestyle{fancyplain}
%\renewcommand{\chaptermark}[1]{\markboth{#1}{bla}}
 
%\lhead{\fancyplain{}{\thepage}}
\chead{Project 14 \textit{, Regeltechniek \& Systeemdynamica}, November 2016}
\lhead{}
\rhead{\thepage /22}
\lfoot{}
\cfoot{}
\rfoot{}

\makeatletter
\@addtoreset{section}{part}
\makeatother

\begin{document}


\begin{table}[ht] 
\begin{tabular}{ M{3cm} M{3.5cm} M{3,5cm} M{3.5cm} N }
2 x 10 =    & 5 x 6 =   & 9 x 2 =    &   5 x 2 =        &\\[1em] 
10 x 6 =    & 9 x 1 =   & 2 x 2 =    &   7 x 2 =        &\\[1em] 
7 x 7 =    & 4 x 10 =   & 5 x 7 =    &   8 x 8 =        &\\[1em] 
5 x 8 =    & 10 x 1 =   & 3 x 7 =    &   8 x 7 =        &\\[1em] 
6 x 6 =    & 1 x 7 =   & 8 x 9 =    &   2 x 3 =        &\\[1em] 
\end{tabular}\end{table} 

\begin{table}[ht] 
\begin{tabular}{ M{3cm} M{3.5cm} M{3,5cm} M{3.5cm} N }
4 x 1 =    & 4 x 7 =   & 1 x 2 =    &   9 x 4 =        &\\[1em] 
10 x 2 =    & 6 x 10 =   & 1 x 3 =    &   5 x 10 =        &\\[1em] 
2 x 7 =    & 9 x 3 =   & 8 x 5 =    &   3 x 1 =        &\\[1em] 
5 x 3 =    & 1 x 8 =   & 4 x 9 =    &   2 x 6 =        &\\[1em] 
2 x 9 =    & 4 x 3 =   & 9 x 6 =    &   4 x 2 =        &\\[1em] 
\end{tabular}\end{table} 

\begin{table}[ht] 
\begin{tabular}{ M{3cm} M{3.5cm} M{3,5cm} M{3.5cm} N }
9 x 9 =    & 7 x 10 =   & 7 x 9 =    &   7 x 8 =        &\\[1em] 
1 x 6 =    & 6 x 5 =   & 10 x 5 =    &   10 x 4 =        &\\[1em] 
3 x 9 =    & 4 x 4 =   & 7 x 1 =    &   9 x 8 =        &\\[1em] 
6 x 3 =    & 3 x 4 =   & 3 x 5 =    &   7 x 6 =        &\\[1em] 
5 x 4 =    & 6 x 2 =   & 7 x 3 =    &   8 x 3 =        &
\end{tabular}\end{table} 

\begin{table}[ht] 
\begin{tabular}{ M{3cm} M{3.5cm} M{3,5cm} M{3.5cm} N }
2 x 5 =    & 1 x 5 =   & 2 x 1 =    &   8 x 4 =        &\\[1em] 
1 x 9 =    & 2 x 8 =   & 5 x 9 =    &   4 x 6 =        &\\[1em] 
3 x 8 =    & 6 x 8 =   & 1 x 4 =    &   6 x 9 =        &\\[1em] 
2 x 4 =    & 8 x 1 =   & 3 x 2 =    &   7 x 4 =        &\\[1em] 
1 x 1 =    & 7 x 5 =   & 9 x 7 =    &   8 x 6 =        &
\end{tabular}\end{table} 

\begin{table}[ht] 
\begin{tabular}{ M{3cm} M{3.5cm} M{3,5cm} M{3.5cm} N }
1 x 10 =    & 10 x 7 =   & 3 x 10 =    &   8 x 2 =        &\\[1em] 
6 x 7 =    & 4 x 8 =   & 10 x 9 =    &   6 x 4 =        &\\[1em] 
5 x 5 =    & 8 x 10 =   & 5 x 1 =    &   6 x 1 =        &\\[1em] 
3 x 6 =    & 10 x 8 =   & 4 x 5 =    &   9 x 10 =        &\\[1em] 
10 x 3 =    & 3 x 3 =   & 9 x 5 =    &   10 x 10 =        &
\end{tabular}\end{table} 
\newpage
\end{document}